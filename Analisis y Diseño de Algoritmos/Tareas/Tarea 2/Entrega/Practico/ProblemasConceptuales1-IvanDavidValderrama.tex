%Tarea1

% --- Clase de archivo ---
% Con este comando se define que tipo de documento vamos a hacer
% en este caso un articulo, en una hoja a4 y con tamanio de fuente 11pt.
\documentclass[a4paper,12pt]{article}
% los tamanios mas utilizados son 10,11 y 12 pt y representan un tamanio de referencia, o sea que al cambiar a 12 pt todos los textos van a aumentar proporcionalmente, al estilo homotecia.


% A partir de aqui se definiran ciertos paquetes o funciones a ser utilizadas
% dentro del documento.


% --- codificacion de archivo ---
% El paquete inputenc sirve para que el compilador pueda interpretar los acentos en espaniol de forma estandar, dependiendo del sistema operativo hay que darle una configuracion diferente:
\usepackage[utf8]{inputenc}
% ----------------------------------


% --- idioma ---
% El paquete babel sirve para separar correctamente las palabras en muchos idiomas, aquií­ esta configurado con espaniol (spanish). Tambien sirve para que el Ií­ndic etenga tií­tulo "Indice" en lugar de "Contents".
\usepackage[spanish]{babel} 
% --------------


% --- Fuentes ---
% El paquete fontenc sirve para controlar las fuentes que utiliza el documento
\usepackage[T1]{fontenc}
% Si dejamos comentado el paquete de abajo utilizaremos la fuente por defecto
% si se quita el comentario se usa sans. Se puede usar times, y otras opciones que se dejan comentadas. 
%\usepackage{sans}
%\usepackage{fbb}
%\usepackage{bera}
%\usepackage{sans}
%\usepackage{times}
%\usepackage{libertine} 


\usepackage{dirtytalk} % Paquete para facilitar la notación de citas con el comando \say{}



% --- Gráficos ---
% El paquete graphicx sirve para controlar figuras con \includegraphics.
\usepackage{graphicx}
% Esta linea indica el lugar (path) en el cual se encuentra la carpeta donde colocamos imágenes, para ahorrar colocarlo en cada
% llamado de una imagen en el documento.
\graphicspath{{./figuras/}}
% Estos paquetes permiten colocar varias figuras en el entorno "figure" como subfiguras (cualquiera de los dos).
\usepackage{float}
\usepackage{subfig} 


% --- Lenguaje matemático ---
% fuentes para escribir sí­mbolos
\usepackage{amsfonts,amssymb,amsthm,amsmath}


% --- Tablas ---
% Paquetes para el manejo de tablas, creación de filas y columnas unidas.
\usepackage{multirow} 
\usepackage{multicol} 
% Control de color en tablas muy versátil.
\usepackage[table]{xcolor}


% --- Hipervínculos ---
% paquete para marcar los hiperví­nculos en i­ndice y referencias
\usepackage{hyperref}
% Para citar referencias  
\usepackage{cite} 
\hypersetup{colorlinks, urlcolor=cyan, citecolor=green, linkcolor=blue} % Pinta con color las referencias
\usepackage[hypcap]{caption} % Las imágenes tienen hiperreferencia y se ven completas y no solo la leyenda. 
% Para hacer hiperreferencias a páginas web
\usepackage{url} 


\usepackage{booktabs} 

% --- Numeración ---
% Paquete que cambia como se representa la numeración de las imágenes, ecuaciones y tablas
\usepackage{chngcntr}
% Ahora se nombran por sección reiniciando el conteo en cada sección
\counterwithin{figure}{section}
\counterwithin{equation}{section}
\counterwithin{table}{section}


% Para agregar al índice las refencias
\usepackage[nottoc,notlot,notlof]{tocbibind} 



% Aqui comienzan los datos del trabajo. El comando \date{\today} asigna la fecha en que se compila como la fecha del trabajo, tambien se puede escibe directamente, ej. \date{5 de setiembre de 2012}.

\title{Tarea 2 Problemas conceptuales} 
\author{%
  Iván David Valderrama Corredor\\ %
  Ingeniería de Sistemas y Ciencias de la Computación\\ %
  Pontificia Universidad Javeriana, Cali}
\date{\today}
% ------------------------

%\pagestyle{empty}
% ====================================


%El paquete colortbl sirve para darle color a las tablas
\usepackage{colortbl}

% Este paquete se utiliza para generar texto de relleno.
\usepackage{blindtext}

% este paquete determina que el texto tenga como fuente normal: times, consultar el eva de latex para mas opciones.
%\usepackage{times}


% ===== Encabezado =====
% esta es una posible configuración para el encabezado. 
%Si se comentan estas dos lineas no habrá encabezado y la numeración de página aparecerá abajo de cada hoja. En la página donde se llame a \maketitle no se coloca encabezado.
\pagestyle{myheadings}
\markright{}
% ======================


%% ===== Ajuste layout pagina =====
% define el ancho del texto en la hoja
\setlength{\textwidth}{155mm}
% define el alto del texto en la hoja
\setlength{\textheight}{210mm}
% los márgenes pueden ser editador con
\oddsidemargin=-.25cm
%% ================================

% --- commandos definidos a gusto del usuario ---
\newcommand{\ds}{\displaystyle}
\def\x{{\bf x}}

\newcommand{\subfigureautorefname}{\figureautorefname}

% -----------------

% =====================================================
% =====================================================
\usepackage{listings}



% =====================================================
% ========  Aca comienza el cuerpo del texto ==========
%
\begin{document}
	
% Se renuevan comandos ya existentes de LaTeX como se desee, en este caso del nombre de tablas y figuras.	
\renewcommand{\tablename}{\bfseries Tabla} % Cambia nombre de tablas
\renewcommand{\figurename}{\bfseries Figura} % Cambia nombre de figuras 
%\newcommand{\subfigureautorefname}{\figureautorefname} % Para que al referenciar una subfigura aparezca "Figura"	
% Se refiere a las tablas y figuras con el comando \autoref{label} para que aparezca referenciado automáticamente el nombre de lo referenciado (Tabla o Figura) continuado por el número de la misma.	
%
% El comando \verb+maketitle+ sirve para escribir la cabecera con los datos del trabajo (título, autor y fecha).
\maketitle

% índice
\tableofcontents

% para separar la carátula del texto introducimos un salto de pagina
\newpage

% definimos una sección con el comando section

\section{Problemas conceptuales}

\subsection{Problema 15-4:Printing Neatly(Cormen et. al. página 405).}
*El texto tiene n palabras, cada un con longitud de caracteres variable l1,l2,l3\\
*Entre palabras hay un solo espacio\\
*Cada linea tiene un maximo de M caracteres\\
*i y j son las palabras que se intentan probar por linea\\
* La formula $M - j + i - \sum_{k=i}^{j}lk$ nos da el espacio sobrente al final de cada linea\\
\\
Proporcione un algoritmo de programación dinámica para imprimir un párrafo de n palabras de forma ordenada en una impresora.\\\\
$Lc(i,j) = \left \{ \begin{matrix} INF & \mbox{extras[i,j] < (i.e., palabra i,...,j no encaja)},
\\ 0 &  \mbox{j = n and extras[i,j] >= 0 (el costo de la ultima linea seria 0)},
\\(extras[i,j])^2 &  \mbox{de otra forma}
\end{matrix}\right.$\\\\
\textbf{Respuesta}\\\\
\lstinputlisting{Punto1.py}
\cite{CLRS}\cite{Tushar}\\\\
Analice los requisitos de tiempo y espacio de ejecución de su algoritmo.\\\\
Tanto el tiempo como el espacio de ejecución del algoritmo son $\Theta(n^2)$
\subsection{Problema 15-9:Breaking a String(Cormen et. al. página 410).}
Diseñe un algoritmo que, dados los números de caracteres después de los cuales se rompa, determine una manera menos costosa de secuenciar esos descansos. Más formalmente, dada una cadena S con n caracteres y una matriz L (1..m) que contiene los puntos de interrupción, calcula el costo más bajo para una secuencia de interrupciones, junto con una secuencia de interrupciones que permite este costo\\\\
El problema es muy similar al problema de "matrix-chain multiplication"\\\clearpage
\textbf{Respuesta}\\\\
\lstinputlisting{Punto2.py}
\cite{CLRS2}\\\\
Dada cada iteración del bucle, mas los bucles internos, el tiempo total de ejecución es $\Theta(m^3)$

\subsection{Ejercicio 9:High Performance Computing(Kleinberg and Tardos página 320).}
\textbf{Respuesta}\\\\
(a)\\\\
\begin{tabular}{lc}
\hline \hline
 Día & \mbox{[1] [2] [3] [4] [5] [6]} \\\hline
 X & \mbox{[10] [9] [8] [7] [9] [8]}\\
 S & \mbox{[9] [8] [7] [6] [3] [1]}\\
\\
\hline \hline
\end{tabular}\\
La mejor solucion sería hacer un reboot el día 4 debido a que el total seria $9 + 8 + 7 + 0 + 9 + 8 = 41$ y sin hacer reboot seria $8 + 6 + 4 + 2 = 34 $.\\\\
(b)
\lstinputlisting{Punto3.py}
El tiempo total de ejecución es $\Theta(n)$
\clearpage
% =====================
\begin{thebibliography}{15}

\bibitem{CLRS}
  CLRS Solutions,
  \emph{15-4 Printing neatly}.\\
  https://walkccc.github.io/CLRS/Chap15/Problems/15-4/

\bibitem{Tushar}
  Tushar Roy,
  \emph{Text Justification Dynamic Programming}.\\
  https://www.youtube.com/watch?v=RORuwHiblPc
  
\bibitem{CLRS2}
  CLRS Solutions,
  \emph{15-4 Printing neatly}.\\
  https://walkccc.github.io/CLRS/Chap15/Problems/15-9/

\bibitem{CS}
  Computer Science,
  \emph{Using dynamic programming to maximize work done}.\\
  https://cs.stackexchange.com/questions/48980/using-dynamic-programming-to-maximize-work-done

\end{thebibliography}

% =====================

\end{document}
